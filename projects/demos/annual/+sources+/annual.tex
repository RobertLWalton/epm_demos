\documentclass[12pt]{article}
\usepackage{times}
\begin{document}
\newcommand{\problem}[1]{\underline{\Large \bf #1}}
\renewcommand{\section}[1]{\bigskip\underline{\bf #1}\\}
\newcommand{\header}[1]{\underline{\bf #1}}
\setlength{\parindent}{0.0in}
\setlength{\parskip}{1ex}

\problem{Annual Return}

If you are considering investing in a mutual fund, you
will look up the fund and find the average annual return
of the fund for the last 1 year, the last 5 years, and
the last 10 years.  However, these are surprisingly
deceptive measures of fund performance.

A much better measure is a matrix that gives the average
annual return $A(y_b,y_s)$ for an investment in the
fund bought in year $y_b$ and sold in year $y_s$, for
all $y_{min}\le y_b\le y_s\le y_{max}$, where
$y_{min}$ and $y_{max}$ are at least 20 years apart.
This matrix answers questions of the form: if I bought
in year $y_b$ and sold in year $y_s$, what would my
average annual return be.

From this you will find that there are some good years
to buy and some bad years to buy, and some good years
to sell and some bad years to sell.  But if the
average annual return is good enough for you for most
purchase years and most sell years, perhaps the fund
is a good investment.

You are being asked to compute such average annual
return matrices.

To compute $A(y_b,y_s)$ we need to know the value $v_b$
of some investment in year $y_b$ and the value $v_s$ of the
\underline{same} investment in year $y_s$.  Then $v_s/v_b$ is the
total return as a ratio, and if $v_s/v_b=B^{(y_s-y_b)}$, then
$B$ is the average annual return as a ratio.  But we
do not want a ratio; we want a percentage.  So we compute
$A(y_s,y_b)$ so that $B=1+A(y_s,y_b)/100$.

\section{Input}
The first line is a title line, that names the fund.
It is at most 80 characters.

After the title line there are $n\le 30$ lines, with the $i$'th
line containing two numbers, $y_i$ and $v_i$ for
$1\le i\le n$.  The $y_i$ are year numbers, and
are integers, $0<y_i\le 9999$, such that $y_{i+1}=y_i+1$ for
$1\le i<n$.  The $v_i$ are value numbers, and
are floating point such that $0<v_i\le 1,000,000$.

The value numbers are the value of some investment
made in the fund before year $y_1$, assuming that
once the investment was made no more money was ever
put into the fund and no money was ever taken out.
More specifically, any money made by the investment
was automatically reinvested in the fund, which is
a typical thing to do with mutual funds.

Its not important how big the initial investment
was or how long ago it was made, as the only thing
we are interested in is ratios $v_j/v_i$ for
$j>i$.

Input ends with an end of file.


\section{Output}
First output the title line, exactly as it is in the
input.

Then output a matrix of $n$ rows each with $n$
columns, with each row on one line, and each column
taking exactly $6$ spaces.

For the first row, its first column is blank, and the
$k+1$'st column contains $y_k$ for $1\le k<n$.

For the $j+1$'st row, $1\le j< n$,
the first column contains $y_{j+1}$,
and the $k+1$'st column contains $A(y_k,y_{j+1})$ for $1\le k\le j$.

All numbers must be right adjusted in their $6$ space wide
column.  The $A$ numbers must have exactly $2$ decimal places.



\begin{center}
\small
\begin{tabular}{l|l}
\begin{minipage}{1.0in}
\small
\header{Sample Input}
\begin{verbatim}
CONSTANT
1990 1.0000
1991 1.0001
1992 1.0002
1993 1.0003
1994 1.0004
1995 1.0005
\end{verbatim}
\end{minipage}\hspace*{0.1in}
&
\hspace*{0.1in}\begin{minipage}{3.5in}
\small
\header{Sample Output}
\begin{verbatim}
CONSTANT
        1990  1991  1992  1993  1994
  1991  0.01 
  1992  0.01  0.01
  1993  0.01  0.01  0.01
  1994  0.01  0.01  0.01  0.01
  1995  0.01  0.01  0.01  0.01  0.01
\end{verbatim}
\end{minipage}
\end{tabular}
\end{center}

\bigskip

\begin{tabular}{ll}
Author:	      & Bob Walton $<$walton@acm.org$>$ \\
Date:         & Sat Jul 11 03:55:25 EDT 2020
\end{tabular}

The authors have placed this problem in the public domain;
they make no warranty and accept no liability for this problem.

\end{document}

